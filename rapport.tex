\documentclass[a4paper,11pt]{report}

\usepackage[french]{babel}              %%% langue française
\usepackage[utf8]{inputenc}             %%% encodage utf8
\usepackage{amsmath,amsfonts,amssymb}   %%% pour les maths
\usepackage{fourier}                    %%% police fourier
\usepackage{graphicx}                   %%% pour inclure des images
\usepackage[dvipsnames]{xcolor}         %%% les couleurs
\usepackage[T1]{fontenc}                %%% pour les mots accentués
\usepackage{subcaption}                 %%% pour les sous-figures
\usepackage[french]{minitoc}            %%% sommaire au début de chaque chapitre
\usepackage{hyperref}                   %%% Options pour la tables des matières
\usepackage[top=1.5cm,bottom=1.5cm,left=1.5cm,right=1.5cm]{geometry}   %%% paquet pour formater la page => géométrie

%\usepackage[Lenny]{fncychap}          %%% Ensemble de thème pour les chapitres   (Lenny, Sonny, Glenn, Conny, Rejne, Bjarne, Bjornstrup )

\usepackage[titletoc]{appendix}         %%% Pour les annexes
\usepackage{epstopdf}                   %%% Pour convertir eps vers pdf
%\usepackage[explicit]{titlesec}
\usepackage{empheq}                     %%% Pour mettre une boite autour des équations
\usepackage{minted}                     %%% Pour insérer du code source

\definecolor{titi}{rgb}{0,0.3,0}        %%% pour definir ses propres couleurs
\definecolor{toto}{RGB}{0,0,185}


%%% Modification de la table des matières couleurs cliquable 
\hypersetup{
colorlinks,
linkcolor=toto,    %%% Couleur des liens internes
urlcolor=toto,     %%% Couleur des hyperliens
citecolor=titi,    %%% Couleur des références de la bibblio
breaklinks=true,   %%% Permet le retour à la ligne pour les liens trop longs
backref=true,      %%% Permet d'ajouter des liens 
pagebackref=true,  %%% dans les bibliographies
hyperindex=true,   %%% Ajoute des liens dans les index.
bookmarks=true
}

\setcounter{tocdepth}{3}   %%% Modifie la profondeur de la table des matières
\begin{document}

\renewcommand{\contentsname}{Sommaire}            % Permet de renommer la table des matières 
\renewcommand{\listfigurename}{Liste des figures} % Permet de renommer la liste des figures
\renewcommand{\listtablename}{Liste des tableaux} % Permet de renommer la liste des tableaux

\begin{titlepage}   %%% Début de la page de garde


\begin{figure}[h]
\begin{minipage}{0.20\textwidth}
\centering
\includegraphics[width=\linewidth]{logos/ensg.png}
\end{minipage}
\hfill
\begin{minipage}{0.20\textwidth}
\centering
\includegraphics[width=\linewidth]{logos/lareg.png}
\end{minipage}
\end{figure}
\noindent
\begin{minipage}{0.25\textwidth}
\begin{flushleft}
\textsc{École nationales des \\
Sciences géopgraphiques}
\end{flushleft}
\end{minipage}
\hfill
\begin{minipage}{0.25\textwidth}
\begin{flushright}
\textsc{Laboratoire de Recherche en Géodésie}
\end{flushright}
\end{minipage}
\vspace{3.0cm}
\begin{center}
\textsc{\Large Rapport de Stage, projet ... }
\end{center}
\begin{center}
\rule{\textwidth}{0.6mm}\\[0.4cm]
{\LARGE \bfseries{Intitulé du stage, projet ...}}\\[0.4cm]
\rule{\textwidth}{0.6mm}\\[1cm]

\includegraphics[width=0.45\linewidth]{images/Debspatiaux.png}\\[1.5cm]
\end{center}

\begin{minipage}{0.25\textwidth}
\begin{flushleft}
Thomas Sandri \\
PPMD
\end{flushleft}
\end{minipage}
\hfill
\begin{minipage}{0.25\textwidth}
\begin{flushright}
Commanditaires \\
Abdel Kader
\end{flushright}
\end{minipage}\\[0.7cm]
\begin{center}
Janvier 2025
\end{center}
\end{titlepage}   %% Fin page de garde

\dominitoc

\tableofcontents

\listoffigures

\listoftables

\chapter*{Introduction}

Le freinage atmosphérique est le principal phénomène physique responsable
de la rentrée des débris spatiaux dans l'atmosphère

\chapter{Frottement atmosphérique}

\minitoc

\section{Définition}

La force de freinage atmosphérique est définie par l'équation \ref{eq:freinage} :

\begin{equation}
\overrightarrow{f}_{atm} = -\frac{1}{2} \, \frac{m \, \rho}{S} \, \frac{\overrightarrow{v}. \overrightarrow{v}}{||\overrightarrow{v}||} \, \overrightarrow{v}
\label{eq:freinage}
\end{equation}

Cette équation nous vient de \cite{king} mais aussi de \cite{deleflie}. Pour plus de détails sur cette formule voir l'annexe \ref{Annex:Un}.
\subsection{Une première sous-section}

Les images \ref{img:Stella 2012} et \ref{img:Stella 2013} de \ref{img:Densité}  montrent l'évolution de la densité atmosphérique pour 

\begin{figure}[h]
\centering
\begin{subfigure}{0.45\textwidth}
\centering
\includegraphics[width=0.9\linewidth]{images/densite/06_06_12.eps}
\caption{2012}
\label{img:Stella 2012}
\end{subfigure}
~
\begin{subfigure}{0.45\textwidth}
\centering
\includegraphics[width=0.9\linewidth]{images/densite/06_06_13.eps}
\caption{2013}
\label{img:Stella 2013}
\end{subfigure}
\caption{Évolution de la densité atmosphérique}
\label{img:Densité}
\end{figure}

\section{Une deuxième section}
Le tableau \ref{tab:Ordre} montre l'effet du freinage atmosphérique sur la période de vie des satellites.
\begin{table}[!h]
\centering
\begin{tabular}{|c|c|}
\hline
200 & 60 jours \\
\hline
250 & 220 jours \\
\hline
500 & quelques années \\
\hline
1000 & plusieurs siècles \\
\hline
1500 & 10000 ans\\
\hline
\end{tabular}
\caption{Ordre de grandeur de la durée de vie moyenne d'un satellite en fonction de son altitude}
\label{tab:Ordre}
\end{table}
\subsubsection{Une première sous sous-section}

\chapter{Modèle de décroissance}
\minitoc
\section{Les satellites}

Des modèles de décroissance sous forme polynomiale ont été établis. La figure \ref{img:décroissance} résume tout ceci.

\begin{figure}[h]
\centering
\begin{subfigure}{0.45\textwidth}
\centering
\includegraphics[width=\linewidth]{images/modeles/modele_stella.eps}
\caption{Stella}
\label{img:Stella}
\end{subfigure}
~
\begin{subfigure}{0.45\textwidth}
\centering
\includegraphics[width=\linewidth]{images/modeles/modele_starlette.eps}
\caption{Starlette}
\label{img:Starlette}
\end{subfigure}

\begin{subfigure}{0.45\textwidth}
\centering
\includegraphics[width=\linewidth]{images/modeles/modele_larets.eps}
\caption{Larets}
\label{img:Larets}
\end{subfigure}
~
\begin{subfigure}{0.45\textwidth}
\centering
\includegraphics[width=\linewidth]{images/modeles/modele_westpack.eps}
\caption{Westpack}
\label{img:Westpack}
\end{subfigure}
\caption{Modèles de décroissance}
\label{img:décroissance}
\end{figure}

Allez une belle équation bien sale avec une belle couleur
{\color{ForestGreen} \large
%\begin{eqnarray}
\begin{empheq}[box=\fbox]{align}
\dot{a}  = & \frac{2 \, a^2}{h} \, \left[e \, \sin v . a_r + \left(1+e \, \cos v \right). \, a_{\theta} \right] \\
\dot{e}  = & \frac{h}{\mu} \, \left[\sin v . a_r + \left(\frac{e+\cos v}{1+e \, \cos v}\right).a_{\theta} \right] \\
\dot{i} = & \frac{r \, \cos \left(v+\omega\right)}{h}. a_h\\
\dot{\omega} = & \frac{h}{e \, \mu} \left[-\cos v.a_r+\frac{2+e\, \cos v}{1+e \, \cos v} \, \sin v.a_{\theta} \right] -\frac{r \, \sin\left(v+\omega\right) \, \cos i}{h \, \sin i}.a_h\\
\dot{\Omega} = & n-\frac{h \, \sqrt{1-e^2}}{\mu} \, \left[\left(\frac{2}{1+e\, \cos v}-\frac{\cos v}{e}\right).a_r+\frac{\sin v}{e}\left(\frac{2+e\, \cos v}{1+e\, \cos v}\right).a_{\theta}\right]\\
 = &  n-\sqrt{1-e^2}\, \left(\frac{2 \, r}{h}; a_r+\dot{\omega}+\dot{\Omega} \, \cos i \right) \nonumber
\end{empheq}
%\end{eqnarray}
}
\begin{appendices}
\chapter{Codes sources}
\inputminted[frame=lines,  %% Sépare le code par des lignes
framesep=2mm,
linenos]{cpp}{codes/test.cpp}
\label{Annex:Un}
\inputminted[frame=lines,framesep=2mm,linenos]{fortran}{codes/test.f90}
\inputminted[frame=lines,framesep=2mm,linenos]{matlab}{codes/test.m}
\end{appendices}

\bibliography{biblio}

\bibliographystyle{apalike}

\end{document}
